\section{现场添加指令和答辩记录}

\subsection{现场添加指令}
\subsubsection{分工情况}
两人一起添加指令。
\subsubsection{完成情况}
修改aludec(添加RELU的alucontrol),maindec(添加RELU的译码),opcodedefines.vh文件中添加RELU的opcode和aludefines.vh文件中添加RELU的alucontrol,在alu中添加RELU操作。添加指令逻辑思路清晰正确,编译通过。
\subsection{现场答辩记录}
\subsubsection{问题1}
\begin{itemize}
    \item 问题:cp0的BadVaddr寄存器是放什么的?
    \item \stunameb 回答:地址错例外的那一条指令的虚地址,地址错例外是pc或lw或sw的时候地址没有对齐导致的。
\end{itemize}
\subsubsection{问题2}
\begin{itemize}
    \item 问题:cp0的epc寄存器是放什么的?
    \item \stunameb 回答:异常处理执行完后的返回地址,如果异常在延迟槽就要返回上一条,否则就是该条。
    \item \stunamea 补充:异常分为异步异常和同步异常,前者的返回地址是例外地址的下一条,后者的返回地址是例外的地址。
\end{itemize}
\subsubsection{问题3}
\begin{itemize}
    \item 问题:AXI是不是用的转接桥?
    \item \stunameb 回答:是的,从mips中的sram\_to\_sram\_like将sram接口转为sram\_like接口,然后经过mmu模块确定是否要过dcache。对于指令来说必须要经过icache,经过了icache miss,进入axi转接桥。对于数据来说,首先根据是否经过dcache分为两条路,一条经过cache,另一条直接传出去。如果cachemiss或者直接传出去,就通过axi转接桥送给外面的组件。
\end{itemize}
\subsubsection{问题4}
\begin{itemize}
    \item 问题:跳转放在了哪个阶段?
    \item \stunameb 回答:Decode阶段。
\end{itemize}
\subsubsection{问题5}
\begin{itemize}
    \item 问题:整个设计的冒险有哪些?
    \item \stunameb 回答:分为数据冒险和控制冒险,对于数据冒险来说,要用到之前没有写回的数据,有的需要stall有的需要前推,stall分为两类:一类是上一条指令还在E阶段,M阶段才能拿到,D阶段娶不到M阶段的值,必须stall一个周期;另一类是我们没有实现E阶段向D阶段的前推,参考lab4的设计,stall了一个周期。前推部分也有两类,一类是W向E前推的,还有一类是M向E和D的前推的。
\end{itemize}
\subsubsection{问题6}
\begin{itemize}
    \item 问题:sram\_like的握手信号是什么逻辑?
    \item \stunameb 回答:时序逻辑。
\end{itemize}
\subsubsection{问题7}
\begin{itemize}
    \item 问题:请你描述一下d\_cache的状态机。
    \item \stunameb 回答:d\_cache状态机分为三个状态,空闲、读ram的RM状态和写ram的WM状态,他们各自转移的逻辑是,当空闲状态遇到读miss时,会进入到RM状态,遇到写时无论是hit还是miss,都要进入WM状态,其他情况保持不变,RM在握手之后转到空闲状态,WM也是,握手是指数据握手。
\end{itemize}
\subsubsection{问题8}
\begin{itemize}
    \item 问题:请问遇到异常flush的时候,div已经进入到除法状态了,这个时候怎么处理呢?
    \item \stunamea 回答:我们的设计中,flush的时候div\_stall不会置位,div运算的结果按理来说会被flush,以后都用不到了。
\end{itemize}
\begin{center}
\end{center}


\subsubsection{问题9}
\begin{itemize}
    \item 问题:请你们各自说说过程中遇到令你影响深刻的问题。
    \item \stunamea 回答:我们发现在除法最后一个周期的时候,会根据op\_data的值改变得到的结果,而op\_data在整个过程中发生了变化,而op\_data是alu的srca2E传进来的,因此是srca2E出了问题,刚开始的srca2E是M阶段前推过来的,但是那个时候我们的div只stall了前三个阶段,因此,在M阶段的指令流走了之后,前推也没了。虽然M阶段的那条指令已经写入了regfile,但是由于stall的存在,不能把正确的值读到E阶段。
    \item \stunameb 回答:axi的sram转类sram的接口中,我们的接口用的是系统结构提供的那个接口,但是我们发现那个接口加上dcache就炸了perfdiff,对着波形图找了很久,最后发现d\_sram\_to\_sram\_like那个地方wen是0000,但是出来的size是10,地址低两位不是00而是10,但1010不是一个合法组合,所以就把低两位直接改成0,就可以过perfdiff了。
\end{itemize}
